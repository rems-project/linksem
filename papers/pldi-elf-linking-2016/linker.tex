\section{Formalising linking}

Building on this formalision of ELF, we can now formalise the operation of a linker,
including the relevant features of linker-speak.
We focus on static linking of executables for the moment.

\subsection{Overview}

Our formalisation takes the form of an executable specification that 
is both a linker and a link checker.

It is designed around the abstraction of \emph{memory images} and associated \emph{annotation} metadata.
Linking is expressed as a transformation of memory images.
Each input ELF file is represented abstractly as a partial memory image, consisting of 
collection of elements, mostly mirroring ELF sections.
Byte ranges within elements are labelled with metadata \emph{tags}, 
mostly mirroring ELF metadata such as symbols, relocations, and section properties.
At the end of the linking process, a single memory image is assembled,
which is transformed back into an ELF file.
%, with its elements
%forming the content and its metadata tags used to generate the symbol table, section headers
%and so on.

The link checker is invoked the same way as the emulated linker, supporting the same command-line options,
but with an output file (named with \textsf{-o}) that already exists.
The checker builds an output memory image---effectively performing its own link---checks consistency with the image
embodied by the actual output file.
(FIXME: also do post-load check: start the program up in a ptrace'd state 
and compare each word of the image?)
Currently the consistency check is a very tight equivalence, down to address assignments and bytewise content
(modulo any bytes marked as ``don't-care'' in the image, such as padding).
This is not inappropriate, since the very simplest link jobs are entirely deterministic.
We discuss which linker features introduce looseness in \S\ref{sec:looseness}.
The comparison also abstracts away from issues of how the output file was rendered into ELF: 
it compares only the embodied memory imag. 
Details such as the ordering of ELF symbols or section headers are not significant.

The notion of memory images generalises to \emph{symbolic} memory images, in which 
each memory image element's address and content may be expressed in terms of 
symbolic variables and unordered concatenations of fragments, rather than precise addresses
and bytes.
This approach potentially allows a family of possible links to proceed at once,
and accommodating the looseness inherent in checking for ``any valid link''.
Currently, however, only a relatively concrete representation of section contents has been specified, 
so the linker must operate concretely for any non-trivial link job.
To accommodate concrete variation among linkers, loose behaviour is factored into 
a family of ``personality functions'', allowing different linkers' choices to be 
captured independently of the core logic. 
At present, only the GNU linker personality is substantially developed.

%a reference implementation optimised for readability
%\and portability

% EXAMPLES of the liftable-out portions of the spec 
% e.g.\ detailing symbol binding.
% 
% a test oracle.
% 
% a basis for reasoning? WHAT can we argue here?
% 

% To allow per-linker divergences, 
% the linker is factored to invoke 
% linker ``personality functions'' corresponding to 
% the loosenesses identified earlier (\S\ref{sec:looseness})
% 
% \begin{itemize}
% 
% \item seen ordering
% 
% \item placing orphans
% 
% \item simplifying relocations
% 
% \item concretising padding etc.
% 
% \item concretising support features (order etc.)
% 
% \item ELF headers, program headers etc.
% 
% \end{itemize}

\subsection{Linking \texttt{hello}}
\label{sec:hello}

In the C programming language, a ``hello, world!'' program is among the simplest possible.
However, for a linker, such a simple program amounts to a very complex job,
since it links with the C library---often the most complex library on the system,
in terms of the linker features it exercises.

In this section we outline what happens when a hello-world C program, compiled with \textsf{gcc},
is linked against uClibc\footnote{\url{http://uclibc.org/}}, 
a fully-featured C library slightly simpler than the system-default GNU C library.
% At each point we describe the relevant features of our formalisation

\paragraph{Parse command line}
This stage is responsible for identifying input files and link options.
The command for linking a hello-world program, slightly simplified to omit directory names, 
is as follows.

\begin{lstlisting}[language=plain,basicstyle=\footnotesize\sffamily]
ld -m elf_x86_64 -static -o hello crt1.o crti.o crtbeginT.o hello.o \
   --start-group -lgcc -lc --end-group crtend.o crtn.o 
\end{lstlisting}

Options beginning \textsf{-lX} say ``look up library \textsf{libX} on the library search path'', 
and denote input files; they resolve to files \textsf{libgcc.a} and \textsf{libc.a}.
Only \textsf{hello.o} came from compiling the program;
the other object files have been supplied by the compiler (\textsf{libgcc.a}, \textsf{crt\{1,i,n\}.o})
and the C library (\textsf{libc.a}, \textsf{crt\{beginT,end\}.o}).
Other options are modifiers to the link; some apply to the whole link 
(like \textsf{-m~elf\_x86\_64}, selecting x86-64 output)
whereas others affect only the input files that follow them, or until negated 
(here \textsf{--start-group} and \textsf{--end-group}).\footnote{Confusingly, \textsf{-static} is 
also of this kind: if any \textsf{-lX} options preceded it, they might be linked dynamically, meaning the output
would \emph{not} be statically linked.}
``Groups'' of archives affect symbol resolution: within a group,
symbols may be resolved cyclically, whereas outside a group, 
references from archives only bind to objects appearing earlier in command-line order.\footnote{This
behaviour dates from a time when unnecessarily scanning archives would create
noticeable link-time slowdowns.}
Many modifiers cancel preceding ones or the relevant defaults 
(such as \textsf{-Ttext} which sets the output text section address), 
but others add content every time they are invoked
(such as \textsf{--defsym}, which defines a new symbol).

The command line formalisation is structured as an interpreter, whose state 
is the collection of input files together with two sets of active modifiers: one for subsequent 
input files, one for the whole link.
It is parameterised by a large list of option definitions, 
each incorporating the option syntax and also its semantics as a function from state to state.
The table (see Fig.~\ref{fig:command-line} resembles the linker's \textsf{--help} text, expanded
with semantics for each option.
Arguments not matching any list entry are treated as input files, cloning the current per-input-file options state.
Complex options such as \textsf{--push-state} necessitate that a state include both a current value and a stack of previously saved values.

\begin{figure*}
\begin{lstlisting}[language=plain,basicstyle=\scriptsize\sffamily]
let command_line_table = [
  (["-o"; "--output"],       (["FILE"], []), fun args -> set_or_replace_opt (OutputFilename(head (fst args))), "Set output file name");
  (["-Bsymbolic-functions"], ([], []),       fun args -> set_or_replace_opt (BindFunctionsEarly), "Bind global function references locally");
  (["-Ttext-segment"],       (["ADDR"], []), fun args -> set_or_replace_opt (TextSegmentStart(parse_addr (head (fst args)))), "Set address of text segment");
  (["-("; "--start-group"],  ([], []),       fun _ ->   (fun state -> start_group state), "Start a group");
  (["-)"; "--end-group"],    ([], []),       fun _ ->   (fun state -> end_group state), "End a group");
  (* ... *)
]
\end{lstlisting}
\caption{Excerpt from the specification of GNU linker command-line options}
\label{fig:command-line}
\end{figure*}

\paragraph{Enumerate objects} 
Although only eight files appear in the command, the two archives contain a total of 891 objects, so 
a total of 897 object files are input to the link.
The linker must now resolve symbol references between these objects.
The semantics of symbol resolution are complex, since 
it matters whether an object file came from an archive, and, if so, 
whether the referencing object appeared to the left or right of that archive on the command line.
A map of all definitions is constructed, grouped by name.
Then the semantics is factored into an ``eligibility function'' answering whether a given
reference can bind to a given definition, and an 
ordering on eligible definitions such that
the first eligible definition is the intended referent.
The ordering is based on command-line order, but also accounts for the semantics of 
substitution: definitions in relocatable files take precedence over archives, hence 
providing the semantics necessary for the \textsf{malloc.o} substitution example \S\ref{sec:substitution}.
Once all symbol references are resolved, any unreferenced objects can be excluded.
In our case, this leaves 61 objects in the link.

% these amount to Having collected a set of input files, the options applying to each,
% and the global link options, 
% we inspect the input files to reveal their structure.
% Each input file might be an object file, an archive or a linker script
% (or, with dynamic linking support, a shared object).
% The semantics of symbol resolution is different in each case,
% so even after enumerating the contents of an archive, we must remember
% which symbol-resolution behaviour is associated with it.
% In practice, for error reporting, we remember all details of the originating archive.
%     let def_is_eligible = (fun (def_idx, def, def_linkable) -> 
%         let ref_is_unnamed = (ref.ref_symname = "")
%         in
%         let ref_is_to_defined_or_common_symbol = ((natural_of_elf64_half ref.ref_syment.elf64_st_shndx) <> stn_undef)
%         in
%         let def_sym_is_ref_sym = (ref_idx = def_idx && ref.ref_sym_scn = def.def_sym_scn
%             && ref.ref_sym_idx = def.def_sym_idx)
%         in
%         let (def_obj, (def_fname, def_blob, def_origin), def_options) = def_linkable
%         in
%         let (def_u, def_coords) = def_origin
%         in
%         let (def_in_group, def_in_archive) = match def_coords with
%               InArchive(aid, aidx, _, _) :: InGroup(gid, gidx) :: [_] -> (Just gid, Just aid)
%             | InArchive(aid, aidx, _, _) :: [_]                       -> (Nothing, Just aid)
%             | InGroup(gid, gidx) :: [_]                              -> (Just gid, Nothing)
%             | [_]                                                    -> (Nothing, Nothing)
%             | _ -> failwith "internal error: didn't understand origin coordinates of definition"
%         end
%         in
%         let ref_is_leftmore = ref_idx <= def_idx
%         in
%         (* For simplicity we include the case of "same archive" in "in group with". *)
%         let ref_is_in_group_with_def = match def_in_group with 
%               Nothing -> false
%             | Just def_gid -> 
%                 match ref_coords with
%                   InArchive(_, _, _, _) :: InGroup(gid, _) :: [_] -> gid = def_gid
%                 | InGroup(gid, _) :: [_]                       -> gid = def_gid
%                 | _ -> false
%                 end
%             end
%         in
%         (* but maybe same archive? *)
%         let ref_and_def_are_in_same_archive = match (def_coords, ref_coords) with
%             (InArchive(x1, _, _, _) :: _, InArchive(x2, _, _, _) :: _) -> x1 = x2
%             | _ -> false
%         end
%         in
%         let def_is_in_archive = match def_in_archive with
%             Just _ -> true
%             | Nothing -> false
%         end
%         in
%         if ref_is_to_defined_or_common_symbol then def_sym_is_ref_sym
%         else 
%             if ref_is_unnamed then false
%             else
%                 if def_is_in_archive
%                 then
%                     (not ref_is_weak) 
%                     && (
%                            ref_is_leftmore
%                         || ref_and_def_are_in_same_archive
%                         || ref_is_in_group_with_def
%                     )
%                 else 
%                     true
%     )

\paragraph{Optimise}   -- simplify relocs etc.. 
This is where we diverge from GNU currently, because of mov-to-lea optimisation.
 

\paragraph{Generate support structures}
The linker is responsible for generating the support structures used by different code models.
In most ABIs, these include the GOT (global offset table; a table of pointers) and PLT 
(procedure linkage table; a table of trampolines).
These are used for indirect addressing, when code compiled with narrow addressing modes
must reach definitions which might be located far away.
The linker remains oblivious to the particulars of the models;
it simply generates the GOT and PLT according to the relocations in the input.
In order to allow these structures to be placed in memory by the linker script, 

%(These relocations implicitly embody a code model, since under a different code model,
%the compiler would have selected different instructions with different relocations.)
However, the linker's address assignment proceeds purely according to the linker script, 
again without reference to any particular code model.
For example, to generate a GOT, the linker simply generates one entry 
or each distinct symbol definition that is used in a GOT-based relocation
(although it is the input code's code model that determines which references these are).\footnote{Mismatch of code models happens when input code uses addressing modes which cannot reach far enough 
to point to the addresses assigned by the linker.
In such cases, the linker's relocation calculations overflow, the overflow is detected
and the link is aborted.
This is most common when the linker is deferring bindings until dynamic link time: 
any deferred binding must be assumed to be potentially located far away,
so 32-bit addressing are not wide enough unless indirected via the GOT or PLT.
On seeing a reference not relocated this way, 
the linker aborts with the infamous ``recompile with \textsf{-fPIC}'' message.}


Note that dynamic relocations, GOT / PLT etc.. must be reified *before* linker script runs, 
so that the script can control where they get placed.
This is a nasty hack in GNU (magic sections appear on the first input file), 
and generally the treatment of this is a kind of looseness (CHECK that it's mentioned
in the main list).

\paragraph{Compose output sections} ---a pass over the linker script, 
      also generating symbol definitions etc. (mention PHASE issue)

% (optional) discard unreferenced sections

\paragraph{Assign addresses} ---another pass over the linker script.
We can't do it at the same time as the previous pass, because\ldots{}

\paragraph{Apply relocations} ---now that addresses have been assigned, 

\paragraph{Generate output} ---use a standard recipe to compute an output ELF file 



\subsection{Looseness}
\label{sec:looseness}

The basic operation of a linker is to concatenate inputs 
into a new, combined output. 
This sounds deterministic, and indeed, linking is deterministic in simple cases.
For example, a link job controlled by a well-defined linker script
and whose input consists only of simple freestanding object files
will be deterministic---unless it contains common symbols, 
orphan sections, section groups, mergeable sections or similar features,
or if the linker is required to insert padding.
In practice, nearly all link jobs have one or other of these properties.

\paragraph{Output ordering}
Archive members are ordered in ``the order in which they are seen during the link''.
This means the order is determined by the linker's algorithm for
traversing the dependency graph. 
For efficiency, 
and to implement the rules for binding to archive members as a side-effect, 
this is typically neither a depth- nor breadth-first traversal of references,
but a traversal of \emph{definitions} in linker command-line order
combined with limited back-tracking 
(re-scanning from the start of the current archive whenever a new archive
member is included).
Of course, alternative implementations are possible, 
so this ordering may only be specified loosely.

\paragraph{Section padding amount}
Alignment constraints may be imposed 
in two ways: when input sections are appended to an output section, 
each with its own alignment; 
and when one output section is ended and another begins.
Both cause padding bytes to be inserted. 
In the former case, the linker chooses the greatest alignment of all input sections
that make up the output section.
In the latter case, the linker inserts padding before beginning a fresh output section.
In general, superfluous padding is not an error, but 
is wasteful.
Our specification outlaws this waste, 
by taking the view that the minimum amount be the only allowable amount.
Unfortunately, we have seen cases where the \textsf{gold} linker inserts
superfluous padding, e.g.\ starting at a 16-byte-aligned address
and adding a whole 16 bytes of padding
before the next section of 16-byte alignment.
It is unclear whether this is a bug; 
we believe so (and have filed GNU binutils bug 18979 (FIXME: remove for blind review)).

\paragraph{Section padding contents}
It is not specified what byte values a linker may use
to pad a section.
In practice, linkers use zeroes for data and either zeroes or nop-sequences for code.
Nop-sequences have the advantage that control can flow between abutting sections
even in the presence of padding; 
it is not uncommon to invoke this behaviour, 
for example 
the GNU toolchain's process initialization and finalization logic 
is split between \textsf{crti.o} and \textsf{crtn.o},
and control flows across the section boundary.
Inserting null bytes between these sections would generate a bad link.
(Fragmenting a single instruction across section boundaries
is also conceivable, and would be broken by any kind of padding, 
although we have never seen it.)
For an \textit{n}-byte nop-sequence, 
many choices of instruction may be available, 
depending on the lengths of available nop instructions.

% However, sequences containing multi-byte instructions may affect program semantics
% on ISAs with variable-length instruction encodings, 
% if the program can be caused to jump into the middle of the sequence,
% in which case they are no longer a nop.
% It is important that any such sequence \emph{starting at any byte}
% has either fall-through or trap semantics
% according to the host ISA.

\paragraph{Segment padding amount} 
The GNU linker features an optimisation
that tries to optimise disk-memory trade-offs.
It seems reasonable to specify that 
wasteful padding should not be inserted, but 
no single trade-off is always appropriate.
LINK to binutils bug 19203 (REMOVE for blind review).

\paragraph{Common symbol placement}
Linker scripts control all common symbols at once.
Some ABIs also define ``large common''.
Control of the ordering of individual common symbols
is not provided.
Link order is the obvious choice, but cannot be assumed: FIXME do we really see this (I think so)?

\paragraph{``Link once'' or section group selection}
Like common symbols, 
sections that are members of section groups compose differently
from ordinary linker inputs:
there can be only one instance in the output, no matter how many times
a section group is repeated in the input.

\paragraph{Linker script}
It is permissible for a linker not to accept control scripts.
In this case, its behaviour is much more hard-coded,
and much more subject to arbitrary variation.
The \textsf{gold} linker does not accept control scripts, 
but its hard-coded behaviour tries to emulate the default control scripts of the GNU linker
(which are themselves programmatically generated at build time).
It is hard to argue that the GNU linker's script is definitive; 
in practice, a linker not accepting a linker script
adheres to looser criteria equating to ``any sensible script''.
In general, the minimum sensible 
behaviour is to preserve (rather than discard) allocatable sections, 
preserve and transform ABI-defined metadata to retain consistency with the output data,
and abstain from inserting arbitrary additional information
except in ABI-prescribed sections (which should be non-allocatable).  (CHECK for exceptions to either of these.)
Of course, custom linker scripts can deviate from these sensible defaults,
so they are not a requirement imposed on all linkers per se.

\paragraph{Orphan section placement}
Sections not matched by the linker script can be placed anywhere
having suitable flags.

\paragraph{Section merging}
``SHF\_MERGE is an optional flag indicating a possible optimization. The link-editor is allowed to perform the optimization, or to ignore the optimization. 
(from the Oracle guide; in Sys V docs?)

\paragraph{Linker-generated structures}
GOT, PLT and other run-time structures

\paragraph{Relaxation}
Conceptually, linkers do not know about instruction encodings.
Relocations describe how to fix up input files' contents at the byte level,
without reference to what fixed-up instructions mean.
However, for optimisation purposes, most linkers do in fact
know a little about instruction encoding.
They use this knowledge to optimise output sections
for size and/or execution speed, by ``relaxing'' contents
at reference points (choosing a shorter calling sequence, say) 
and section boundaries (overlapping leading and trailing padding in exception
handling information, say).

\paragraph{Phase anomalies} 
A linker necessarily makes multiple passes over its inputs.
Passes include enumerating inputs, calculating output section layout, 
calculating addresses, applying relocations, and so on.
Some linker features interact in ways which induce circular dependencies between
these passes. 
Linkers resolve these in arbitrary and often undocumented ways.
One example is input enumeration: to identify which archive members
to include in a link, the linker searches archives for definitions
of the symbols required by other input objects.
However, the linker script might subsequently provide its own definition
for such a symbol, obviating the need for definitions that were (perhaps transitively) pulled in.
Whether they are removed from the link is an undocumented detail; 
in all linkers we know of, they are not. (FIXME: check this.)
However, linker phases could be restructured so that they are never pulled in at all.
Another example is in linker-generated structures like the global offset table (GOT). 
The memory placement of the GOT is controllable in the linker script,
meaning it must be sized when the script's interpreter runs. 
The same definition-obviation behaviour might remove all via-GOT references
to a given symbol, hence obviating the need for its slot, meaning the size
should be recomputed after script processing, meaning the address-assignment phase
should be re-run. 
Again, real linkers do not do this recomputation.
If they did, it need not converge to a fixed point, since linker script behaviour
may depend on section sizes and hence create an oscillation.
(FIXME: I thik it might also create a via-GOT reference
to a previously non-GOTted symbol, which would make life interesting.
Not yet sure if linker scripts can set up such a reference though.)


